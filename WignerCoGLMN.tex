\documentclass[12pt]{article}
%
\usepackage{amssymb,amsmath,cite}

\setlength{\textheight}{23cm}
\setlength{\textwidth}{16cm}
\setlength{\topmargin}{0cm}
\setlength{\headheight}{0pt}
\setlength{\oddsidemargin}{0pt}
\setlength{\evensidemargin}{0pt}
%\setlength{\unitlength}{0.7mm}
%
\def\nn{\nonumber}

\newtheorem{lemma}{Lemma}%[section]
\newtheorem{prop}[lemma]{Proposition}
%\newtheorem{thm}[lemma]{Theorem}
\newtheorem{thm}{Theorem}
\newtheorem{cor}[lemma]{Corollary}
\newtheorem{defo}{Definition}
%
%
%
\begin{document}

\begin{center}
{\large\bf Reduced Wigner coefficients for unitary representations
of the Lie superalgebra $gl(m|n)$}\\
~~\\

{\large Jason L. Werry, Phillip S. Isaac and Mark D. Gould}\\
~~\\

School of Mathematics and Physics, The University of Queensland, St Lucia QLD 4072, Australia.
\end{center}

\begin{abstract}
In this paper all fundamental Wigner coefficients are determined
algebraically by the eigenvalues of certain generalized Casimir invariants. Here the
method is applied in the context of both type 1 and type 2 unitary representations of the Lie superalgebra $gl(m|n)$.
\end{abstract}

% \begin{flushright}
% May 9, 2012
% \end{flushright}
% 
% \begin{tableofcontents}
% \end{tableofcontents}

\section{Introduction}
In the previous publications \cite{GIW2,GIW3} the matrix elements of unitary representations of $gl(m|n)$ were given explicitly. The resulting closed-form expressions were obtained by utilizing the factorization of a matrix element into a Wigner coefficient and a reduced matrix element. The vector Wigner coefficients thus obtained allow us in this paper to obtain all fundamental $gl(m|n)$ Wigner coefficients (WCs) in the Gelfand-Tsetlin (GT) basis for both type 1 and type 2 unitary representations.
 
A summary of the background regarding the characteristic identities and their associated invariants in Section \ref{prelim}. We then continue by introducing the Wigner coefficients in Section \ref{Wigner} and give their expressions in terms of representation labels of the representation concerned. Finally, Section 4 concludes with a discussion of these results.  

%%%%%%%%%%%%%%%%%%%%%%%%%%%%%%%%%%%%%%%%%%%%%%%%%%%%%%%%%%%%%%%%%%%%%%%%%%%%%%%%%%%%%%%
%%%%%%%%%%%%%%%%%%%%%%%%%%%%%%%%%%%%%%%%%%%%%%%%%%%%%%%%%%%%%%%%%%%%%%%%%%%%%%%%%%%%%%%

\section{Characteristic identities and associated invariants}
\label{prelim}

We utilise the notation used in the series \cite{GIW1,GIW2,GIW3}. The generators of the Lie superalgebra $gl(m|n)$ are denoted $E_{ij}$ where $1 \leq p,q \leq m+n$. The values $1 \leq i,j \leq m$ are called \textit{even} while the values $m < i,j \leq m+n$ are called \textit{odd}.

The graded index notation will be used where the Latin indices $i,j,...$ are used for even quantities and the Greek indices $\mu, \nu, ...$ for odd quantities. The grading operator $(~)$ will then give
$$
(i) = 0,~(\mu) = 1
$$
and then set of generators is then given by 
$$
E_{ij}, E_{i\mu}, E_{\mu i}, E_{\mu\nu}.
$$
We reserve the indices $p,q$ to be ungraded and to range fully from $1$ to $m+n$.

The adjoint tensor operator $\bar{\mathcal{A}}$ constructed in \cite{GIW1} plays an important role in what follows and is defined as the $m+n$ square matrix with entries
$$
\bar{\mathcal{A}}_{pq} = -(-1)^{(p)(q)} E_{qp}.
$$
When $\bar{\mathcal{A}}$ acts on an irreducible $gl(m|n)$ module $V(\Lambda)$ of highest weight $\Lambda$ it will satisfy the characteristic identity
\begin{align}
\prod_{i=1}^m ({\cal \bar{A}} - \bar{\alpha}_i) \prod^n_{\mu = 1} ({\cal \bar{A}} - \bar{\alpha}_\mu) = 0
\label{CharIdent1}
\end{align}
where the adjoint roots $\bar{\alpha}_i, \bar{\alpha}_\mu$ are given in terms of the highest weight labels
$$
\Lambda = (\Lambda_{i=1},...,\Lambda_{i=m}|\Lambda_{\mu=1},...\Lambda_{\mu=n})
$$
as
$$
{\bar{\alpha}}_i= i - 1 -\Lambda_i ,~~ {\bar{\alpha}}_\mu = \Lambda_\mu + m + 1 - \mu. 
$$
Immediately from the characteristic identity, we see that for each integer $r$ where $1 \leq  r \leq m+n$ there exists a projection operator 
$$
\bar{P}[r]: V(\varepsilon_1) \otimes V(\Lambda) \longrightarrow V(\Lambda + \epsilon_r)
$$
given by
\begin{align*}
\bar{P}[r] = \prod^{m+n}_{k \neq r} \left( \frac{{\cal \bar{A}} - \bar{\alpha}_k}{\bar{\alpha}_r
- \bar{\alpha}_k} \right)\nn.
\end{align*}
Similarly we have the vector matrix $\mathcal{A}$ with entries
$$
\mathcal{A}_{pq} = -(-1)^{(p)} E_{pq}.
$$
that satisfy the polynomial identities
\begin{align}
\prod_{i=1}^m ({\cal A} - \alpha_i) \prod^n_{\mu = 1} ({\cal A} - \alpha_\mu) = 0
\label{CharIdent2}
\end{align}
where
$$
{\alpha}_i= \Lambda_i + m - n - i ,~~ {\alpha}_\mu = \mu - \Lambda_\mu -n. 
$$
The associated projection operator 
$$
P[r]: V^*(\varepsilon_1) \otimes V(\Lambda) \longrightarrow V(\Lambda - \epsilon_r)
$$
is then given by
\begin{align*}
P[r] = \prod^{m+n}_{k \neq r} \left( \frac{{\cal A} - \alpha_k}{\alpha_r
- \alpha_k} \right)\nn.
\end{align*}

We will now show that the eigenvalues of the invariants 
\begin{align*}
\bar{c}_r = \bar{P}[r]_{m+n}^{\ m+n}
\end{align*}
and 
\begin{align*}
c_r = P[r]_{\ m+n}^{m+n}
\end{align*}
are essentially squares of reduced Wigner coefficients.


Let $e_i$ denote the Gelfand-Tsetlin (GT) basis states for the vector module $V(\varepsilon_1)$ and ${e^\Lambda_\beta}$ denote the (GT) basis states for the irreducible module $V(\Lambda)$ of highest weight $\Lambda$. Then the matrix elements of $\bar{P}[r]$ can be given in the form
\begin{align}
\left\langle e^\Lambda_\beta | \bar{P}[r]^j_i | e^{\Lambda}_\alpha \right\rangle =
\sum_\gamma \left\langle e^\Lambda_\beta \otimes e_i | e^{\Lambda + \varepsilon_r}_\gamma \right\rangle
\left\langle e^{\Lambda+\varepsilon_r}_\gamma | e_j \otimes e^\Lambda_\alpha \right\rangle, \label{BarPij}
\end{align}
where
$$
\left\langle e^{\Lambda+\varepsilon_r}_\gamma | e_j \otimes e^\Lambda_\alpha \right\rangle,
$$
are the vector (fundamental) Wigner coefficients.
Similarly by denoting $\bar{e}_i$ to be the Gelfand-Tsetlin (GT) basis states of the dual vector module we have 
\begin{align}
\left\langle e^\Lambda_\beta | P[r]^j_i | e^{\Lambda}_\alpha \right\rangle =
\sum_\gamma \left\langle e^\Lambda_\beta \otimes \bar{e}_i | e^{\Lambda - \varepsilon_r}_\gamma \right\rangle
\left\langle e^{\Lambda-\varepsilon_r}_\gamma | \bar{e}_j \otimes e^\Lambda_\alpha \right\rangle, \label{Pij}
\end{align}
where
$$
\left\langle e^{\Lambda-\varepsilon_r}_\gamma | \bar{e}_j \otimes e^\Lambda_\alpha \right\rangle,
$$
are the dual fundamental Wigner coefficients.

From Schur's lemma we observe that the fundamental WCs factorize as follows
\begin{align}
\left\langle\left. 
\begin{array}{c} \Lambda+\varepsilon_k\\ \lambda+\varepsilon_{0_r} \\ {[\Lambda'_0]} \end{array}
\right|\right.
\left.
e_i\otimes \begin{array}{c} \Lambda \\ \lambda \\
{[\Lambda_0]} \end{array}
\right\rangle &= 
\left\langle\left. 
\begin{array}{c} \Lambda+\varepsilon_k\\ \lambda+\varepsilon_{0_r} 
 \end{array}
\right|\right.
\left.
\begin{array}{c} \varepsilon_1 \\ \varepsilon_{0_1} 
 \end{array}
; \begin{array}{c} \Lambda \\ \lambda
 \end{array}
\right\rangle 
\left\langle\left. 
\begin{array}{c} \lambda+\varepsilon_{0_r} \\ {[\Lambda'_0]} \end{array}
\right|\right.
\left.
e_i\otimes \begin{array}{c} \lambda \\
{[\Lambda_0]} \end{array}
\right\rangle,\label{Schur1}\\ 
~~~ \nn\\
\left\langle\left. 
\begin{array}{c} \Lambda-\varepsilon_k\\ \lambda-\varepsilon_{0_r} \\ {[\Lambda'_0]} \end{array}
\right|\right.
\left.
\bar{e}_i\otimes \begin{array}{c} \Lambda \\ \lambda \\
{[\Lambda_0]} \end{array}
\right\rangle &= 
\left\langle\left. 
\begin{array}{c} \Lambda-\varepsilon_k\\ \lambda-\varepsilon_{0_r} 
 \end{array}
\right|\right.
\left.
\begin{array}{c} \bar{\varepsilon}_1 \\ \bar{\varepsilon}_{0_1} 
 \end{array}
; \begin{array}{c} \Lambda \\ \lambda
 \end{array}
\right\rangle 
\left\langle\left. 
\begin{array}{c} \lambda-\varepsilon_{0_r} \\ {[\Lambda'_0]} \end{array}
\right|\right.
\left.
\bar{e}_i\otimes \begin{array}{c} \lambda \\
{[\Lambda_0]} \end{array}
\right\rangle,\label{Schur2}\\
~~\nn\\
1 < i \leq m+n, \nn
\end{align}
where $1 < k \leq m+n+1$,~$1 < r \leq m+n$, 
$\Lambda$ denotes the highest weight of $gl(m|n+1)$, $\lambda$ denotes the highest weight of $gl(m|n)$ and $[\Lambda_0]$ denotes the GT pattern of the $gl(m|n-1)$ subalgebra.

Setting $i=m+n$ gives
\begin{align}
\left\langle\left. 
\begin{array}{c} \Lambda+\varepsilon_k\\ \lambda \\ {[\Lambda'_0]} \end{array}
\right|\right.
\left.
e_{m+n} \otimes \begin{array}{c} \Lambda \\ \lambda \\
{[\Lambda_0]} \end{array}
\right\rangle = \delta_{[\Lambda'_0][\Lambda]}
 \left\langle\left. 
\begin{array}{c} \Lambda+\varepsilon_k\\ \lambda \end{array}
\right|\right.
\left.
\begin{array}{c} 
\varepsilon_1 \\ \dot{0}
\end{array}
;
\begin{array}{c} \Lambda \\
\lambda \end{array}
\right\rangle  , \label{WCBraKet1}
\end{align}
and
\begin{align}
\left\langle\left. 
\begin{array}{c} \Lambda-\varepsilon_k\\ \lambda \\ {[\Lambda'_0]} \end{array}
\right|\right.
\left.
\bar{e}_{m+n} \otimes \begin{array}{c} \Lambda \\ \lambda \\
{[\Lambda_0]} \end{array}
\right\rangle = \delta_{[\Lambda'_0][\Lambda]}
 \left\langle\left. 
\begin{array}{c} \Lambda-\varepsilon_k\\ \lambda \end{array}
\right|\right.
\left.
\begin{array}{c} 
\bar{\varepsilon}_1 \\ \dot{0}
\end{array}
;
\begin{array}{c} \Lambda \\
\lambda \end{array}
\right\rangle  , \label{WCBraKet2}
\end{align}
The WCs in equations (\ref{WCBraKet1}) and (\ref{WCBraKet2}) are given by the eigenvalues of the invariants
$$ 
\bar{c}_r = \bar{P}[r]_{m+n}^{\ m+n}, ~~c_r = P[r]_{\ m+n}^{m+n}
$$
since from equations (\ref{BarPij}) and (\ref{Pij}) we have



\begin{align}
\bar{c}_r  &= \left| \left\langle\left. 
\begin{array}{c} \Lambda+\varepsilon_r\\ \lambda \end{array}
\right|\right.
\left.
\begin{array}{c} \varepsilon_1 \\
\dot{0} \end{array} 
;
 \begin{array}{c} \Lambda \\
\lambda \end{array}
\right\rangle \right|^2, \\
c_r &= \left| \left\langle\left. 
\begin{array}{c} \Lambda-\varepsilon_r\\ \lambda \end{array}
\right|\right.
\left.
\begin{array}{c} \bar{\varepsilon}_1 \\
\dot{0} \end{array}  ; \begin{array}{c} \Lambda \\
\lambda \end{array}
\right\rangle \nn \right|^2.
\end{align}
%where $\bar{e}_r$ denote the Gelfand-Tsetlin (GT) basis states of the adjoint vector module.


%%%%%%%%%%%%%%%%%%%%%%%%%%%%%%%%%%%%%%%%%%%%%%%%%%%%%%%%%%%%%%%%%%%%

\section{Reduced Wigner coefficients} 
\label{Wigner}
The above derivations may be repeated for the $gl(m|n+1)$ case so that we have tensor operators given by the $m+n+1$ square matrices
$$
\bar{\mathcal{B}}_{pq} = -(-1)^{(p)(q)} E_{qp}, ~~~~~\mathcal{B}_{pq} = -(-1)^{(p)} E_{pq}.
$$
that satisfy the usual polynomial identities 
\begin{align*}
\prod_{i=1}^m ({\cal \bar{B}} - \bar{\beta}_i) \prod^{n+1}_{\mu = 1} ({\cal \bar{B}} - \bar{\beta}_\mu) &= 0 \\
	\prod^m_{i=1} ({\cal B} - \beta_i) \prod^{n+1}_{\mu=1} ({\cal B} - \beta_\mu) &= 0 
\end{align*}
with roots given by
\begin{align*}
\bar{\beta}_i  &= i - 1 -{\tilde{\Lambda}}_i , &1\leq i\leq m, \\
\bar{\beta}_\mu &= {\tilde{\Lambda}}_\mu + m + 1 - \mu,  &1\leq \mu \leq n + 1, \\
\beta_i &= {\tilde{\Lambda}}_i + m - n - 1 - i, &1\leq i\leq m, \\
\beta_\mu &= \mu-{\tilde{\Lambda}}_\mu - n - 1, &1\leq \mu \leq n + 1.  
\end{align*}
Similarly, the $gl(m|n+1)$ projection operators are given by
\begin{align}
\bar{Q}[r] = \prod_{k\neq r}^{m+n+1}\left( \frac{{\cal \bar{B}}-\bar{\beta}_k}{\bar{\beta}_r -
\bar{\beta}_k} \right),
\ \ 
Q[r] = \prod_{k\neq r}^{m+n+1}\left( \frac{
{\cal B}-\beta_k}{\beta_r-\beta_k} \right),
\end{align}

The betweenness conditions imply 
\cite{GIW1}, for $1\leq i\leq m$, that we have only two cases
\begin{align}
\beta_i = \left\{ \begin{array}{rl} \alpha_i,& \tilde{\Lambda}_i = 1+\Lambda_i\\
                                    \alpha_i - 1,& \tilde{\Lambda}_i = \Lambda_i 
\end{array} \right.
\nn
%\label{equ5.4}
\end{align}
We therefore define the following index sets
\begin{align}
I_0 &=  \{ 1\leq i\leq m\ |\ \alpha_i=\beta_i\},\nn\\
\bar{I}_0 &=  \{ 1\leq i\leq m\ |\ \alpha_i=1+\beta_i\},\nn\\
I_1 &= \{ 1\leq\mu\leq n\},\nn\\
I &= I_0\cup I_1,\nn\\
I'&= \bar{I}_0\cup I_1,\nn\\
\tilde{I} &= I\cup \{m+n+1\},\nn\\
\tilde{I}' &= I'\cup \{m+n+1\}.
\label{DefIndexSets} 
\end{align}

By considering the characteristic identities satisfied by the invariants $\bar{\mathcal{B}}$ and $\bar{\mathcal{A}}$ we may obtain the invariant $\bar{c}_r$ as a rational polynomial in terms of the roots $\bar{\beta}_i,\bar{\beta}_\mu$ and $\bar{\alpha}_i,\bar{\alpha}_\mu$. Specifically, we have \cite{GIW1}
\begin{align}
\bar{c}_r = \prod_{k\in \tilde{I}',k\neq r} \left(\bar{\beta}_r - \bar{\beta}_k\right)^{-1}\prod_{k\in
I'} \left(\bar{\beta}_r - \bar{\alpha}_k - (-1)^{(k)}\right),\ \ r\in \tilde{I}',
\end{align}
and
\begin{align}
c_r = \prod_{k\in \tilde{I},k\neq r} \left(\beta_r - \beta_k \right)^{-1}\prod_{k\in
I} \left(\beta_r - \alpha_k - (-1)^{(k)}\right),\ \ r\in \tilde{I}
\end{align}

The definitions of the projections $\bar{Q}[r]$ and $\bar{P}[r]$ allow the calculation of the invariant $\bar{\rho}_{ru}$ in the following expressions \cite{GIW2}
\begin{align}
(\bar{P}[u]\bar{Q}[r]\bar{P}[u])_p^{\ q} &= \bar{\rho}_{ru}\bar{P}[u]_p^{\ q},\\
({P}[u]{Q}[r]{P}[u])_p^{\ q} &= {\rho}_{ru}{P}[u]_p^{\ q}
\end{align}
where $\bar{\rho}_{ru}$ and ${\rho}_{ru}$ are $gl(m|n)$ invariant operators given by
\begin{align}
\bar{\rho}_{ru} &= (\bar{\beta}_r-\bar{\alpha}_u +
1)^{-1}(\bar{\beta}_r-\bar{\alpha}_u)^{-1}\bar{c}_r\bar{\delta}_u, \ \ \ (u) = 1, \nn\\
\bar{\rho}_{ru} &= (\bar{\beta}_r-\bar{\alpha}_u +
1)^{-1}(\bar{\beta}_r-\bar{\alpha}_u)^{-1}\bar{c}_r\bar{\delta}_u \ \ \ (u) = 0, u \neq r,
\nn\\
\bar{\rho}_{uu} &= \bar{c}_u \bar{\delta}_u, \ \ \ (u) = 0, \nn\\
\bar{\rho}_{ru} &= (\bar{\beta}_r-\bar{\alpha}_u -
1)^{-1}(\bar{\beta}_r-\bar{\alpha}_u)^{-1}\bar{c}_r \bar{\delta}_u \ \ \ gl(m)
\hbox{~case,} 
\label{RhoBarExp}
\end{align}
and
\begin{align}
{\rho}_{ru} &= ({\beta}_r-{\alpha}_u +
1)^{-1}({\beta}_r-{\alpha}_u)^{-1}{c}_r{\delta}_u, \ \ \ (u) = 1, \nn\\
{\rho}_{ru} &= ({\beta}_r-{\alpha}_u +
1)^{-1}({\beta}_r-{\alpha}_u)^{-1}{c}_r{\delta}_u \ \ \ (u) = 0, u \neq r,
\nn\\
{\rho}_{uu} &= {c}_u {\delta}_u, \ \ \ (u) = 0, \nn\\
{\rho}_{ru} &= ({\beta}_r-{\alpha}_u -
1)^{-1}({\beta}_r-{\alpha}_u)^{-1}{c}_r {\delta}_u \ \ \ gl(m)
\hbox{~case,} 
\label{RhoExp}
\end{align}
whose eigenvalues determine the square of $gl(m|n+1):gl(m|n)$ reduced vector Wigner
coefficients via
\begin{align}
\bar{\rho}_{ru}  &= \left| \left\langle\left. 
\begin{array}{c} \Lambda+\varepsilon_r\\ \lambda+\varepsilon_{0_u} \end{array}
\right|\right.
\left.
\begin{array}{c} \varepsilon_1 \\
\varepsilon_{0_1} \end{array} 
;
 \begin{array}{c} \Lambda \\
\lambda \end{array}
\right\rangle \right|^2, \\
\rho_{ru} &= \left| \left\langle\left. 
\begin{array}{c} \Lambda-\varepsilon_r\\ \lambda-\varepsilon_{0_u} \end{array}
\right|\right.
\left.
\begin{array}{c} \bar{\varepsilon}_1 \\
\bar{\varepsilon}_{0_1}  \end{array}  ; \begin{array}{c} \Lambda \\
\lambda \end{array}
\right\rangle \nn \right|^2,
\end{align}
and where
\begin{align}
\bar{\delta}_u &= (-1)^{|I'|} \prod_{k\in I',k\neq u}\left(\bar{\alpha}_u - \bar{\alpha}_k -
(-1)^{(s)}\right)^{-1} \prod_{k\in \tilde{I}'}\left(\bar{\beta}_k - \bar{\alpha}_u \right), u\in I, \\
\delta_u &= (-1)^{|I|} \prod_{k\in I,k\neq u} \left(\alpha_u - \alpha_k -
(-1)^{(k)}\right)^{-1}\prod_{k\in\tilde{I}} \left(\beta_k - \alpha_u \right),\ \ u\in I',
\end{align}
are the reduced matrix elements.
Note that $\bar{\rho}_{ru}$ in equation (\ref{RhoBarExp}) is non-vanishing only when $r\in \tilde{I}'$ and $u\in I$ while ${\rho}_{ru}$ in equation (\ref{RhoExp}) is non-vanishing only when $r\in \tilde{I}$ and $u\in I'$.

Substituting the expressions for ${c}_r$ and ${\delta}_u$ into equation 
 (\ref{RhoExp}) initially gives
\begin{align*}
{\rho}_{ru} &= ({\beta}_r-{\alpha}_u + 1)^{-1}({\beta}_r-{\alpha}_u)^{-1} \prod_{k\in \tilde{I},k\neq r} \left(\beta_r - \beta_k \right)^{-1}\prod_{k\in
I} \left(\beta_r - \alpha_k - (-1)^{(k)}\right) \\
& ~\times (-1)^{|I|} \prod_{k\in I,k\neq u} \left(\alpha_u - \alpha_k -
(-1)^{(k)}\right)^{-1}\prod_{k\in\tilde{I}} \left(\beta_k - \alpha_u \right),\ \ r\in \tilde{I},u\in I' 
\end{align*}
Now $({\beta}_r-{\alpha}_u)^{-1}$ will cancel the corresponding term from $\delta_u$ while $
({\beta}_r-{\alpha}_u + 1)^{-1}$ will only cancel a term in $c_r$ when $u$ is odd. We therefore have
\begin{align*}
{\rho}_{ru} &= ({\beta}_r-{\alpha}_u + 1)^{-1}\prod_{k\in \tilde{I},k\neq r} \left(\beta_r - \beta_k \right)^{-1}\prod_{k\in
I} \left(\beta_r - \alpha_k - (-1)^{(k)}\right) \\
& ~\times (-1)^{|I|} \prod_{k\in I,k\neq u} \left(\alpha_u - \alpha_k -
(-1)^{(k)}\right)^{-1}\prod_{k\in\tilde{I},k \neq r} \left(\beta_k - \alpha_u \right),\ \ r\in \tilde{I},u\in I' 
\end{align*}
for $u$ even and
\begin{align*}
{\rho}_{ru} &= (-1)^{|I|} \prod_{k\in \tilde{I},k\neq r} 
\left(
\frac 
{\beta_k - \alpha_u  }
{\beta_r - \beta_k }
\right)
\prod_{k\in I,k \neq u} 
\left(
\frac
{ \beta_r - \alpha_k - (-1)^{(k)} }
{\alpha_u - \alpha_k - (-1)^{(k)} }
\right)
,\ \ r\in \tilde{I},u\in I_1
\end{align*}
for $u$ odd.

In the same manner, substituting the expressions for $\bar{c}_r$ and $\bar{\delta}_u$ into equation 
 (\ref{RhoBarExp}) gives
\begin{align*}
\bar{\rho}_{ru} &= (\bar{\beta}_r-\bar{\alpha}_u + 1)^{-1}\prod_{k\in \tilde{I'},k\neq r} \left(\bar{\beta}_r - \bar{\beta}_k \right)^{-1}\prod_{k\in
I'} \left(\bar{\beta}_r - \bar{\alpha}_k - (-1)^{(k)}\right) \\
& ~\times (-1)^{|I'|} \prod_{k\in I',k\neq u} \left(\bar{\alpha}_u - \bar{\alpha}_k -
(-1)^{(k)}\right)^{-1}\prod_{k\in\tilde{I'},k \neq r} \left(\bar{\beta}_k - \bar{\alpha}_u \right),\ \ r\in \tilde{I'},u\in I 
\end{align*}
for $u$ even and
\begin{align*}
\bar{\rho}_{ru} &= (-1)^{|I'|} \prod_{k\in \tilde{I'},k\neq r} 
\left(
\frac 
{\bar{\beta}_k - \bar{\alpha}_u  }
{\bar{\beta}_r - \bar{\beta}_k }
\right)
\prod_{k\in I',k \neq u} 
\left(
\frac
{ \bar{\beta}_r - \bar{\alpha}_k - (-1)^{(k)} }
{\bar{\alpha}_u - \bar{\alpha}_k - (-1)^{(k)} }
\right)
,\ \ r\in \tilde{I'},u\in I_1
\end{align*}
for $u$ odd.

The expressions for the RWCs of equations (\ref{Schur1}) and (\ref{Schur2}) together with their phases \cite{GIW2,GIW3} are then given by
\begin{align*}
\left\langle\left. 
\begin{array}{c} \Lambda+\varepsilon_r\\ \lambda+\varepsilon_{0_u} 
 \end{array}
\right|\right.
\left.
\begin{array}{c} \varepsilon_1 \\ \varepsilon_{0_1} 
 \end{array}
; \begin{array}{c} \Lambda \\ \lambda
 \end{array}
\right\rangle &= (-1)^{(r)(u)} S(r-u) (\bar{\rho}_{ru})^{1/2}
,\\
\left\langle\left. 
\begin{array}{c} \Lambda-\varepsilon_r\\ \lambda-\varepsilon_{0_u} 
 \end{array}
\right|\right.
\left.
\begin{array}{c} \bar{\varepsilon}_1 \\ \bar{\varepsilon}_{0_1} 
 \end{array}
; \begin{array}{c} \Lambda \\ \lambda
 \end{array}
\right\rangle &= (-1)^{(r)(u) +(r) + (u)} S(r-u) ({\rho_{ru}})^{1/2}
\end{align*}
where odd indices are considered greater than even indices and
$$
S(x) = sgn(x),~~S(0)=1.
$$
























%%%%%%%%%%%%%%%%%%%%%%%%%%%%%%%%%%%%%%%%%%%%%%%%%%%%%%%%%%%%%%%%%%%%%%%%%%%%%%%%%
%%%%%%%%%%%%%%%%%%%%%%%%%%%%%%%%%%%%%%%%%%%%%%%%%%%%%%%%%%%%%%%%%%%%%%%%%%%%%%%%%
%\newpage
%Substituting the expressions for $\bar{c}_r$ and $\bar{\delta}_u$ into equation 
% (\ref{RhoBarExp}) initially gives
%\begin{align*}
% \bar{\rho}_{ru} &= (-1)^{|I'|} (\bar{\beta}_r-\bar{\alpha}_u +
%1)^{-1}(\bar{\beta}_r-\bar{\alpha}_u)^{-1} \prod_{k\in \tilde{I}',k\neq r} \left(\bar{\beta}_r - \bar{\beta}_k\right)^{-1}\prod_{k\in
%I'} \left(\bar{\beta}_r - \bar{\alpha}_k - (-1)^{(k)}\right) \\
%& ~\times \prod_{k\in I',k\neq u}\left(\bar{\alpha}_u - \bar{\alpha}_s -
%(-1)^{(s)}\right)^{-1} \prod_{k\in \tilde{I}'}\left(\bar{\beta}_k - \bar{\alpha}_u \right) 
%\end{align*}
%so
%\begin{align*}
% \bar{\rho}_{ru} &= (-1)^{|I'|} (\bar{\beta}_r-\bar{\alpha}_u +
%1)^{-1}\prod_{k\in \tilde{I}',k\neq r} \left(\bar{\beta}_r - \bar{\beta}_k\right)^{-1}\prod_{k\in
%I'} \left(\bar{\beta}_r - \bar{\alpha}_k - (-1)^{(k)}\right) \\
%& ~\times \prod_{k\in I',k\neq u}\left(\bar{\alpha}_u - \bar{\alpha}_k -
%(-1)^{(k)}\right)^{-1} \prod_{k\in \tilde{I}', k \neq r}\left(\bar{\beta}_k - \bar{\alpha}_u \right) 
%\end{align*}
%Now for $(u)=1$ there will be a corresponding term in
%$$
%\prod_{k\in
%I'} \left(\bar{\beta}_r - \bar{\alpha}_k - (-1)^{(k)}\right)
%$$
%that matches
%$$
%(\bar{\beta}_r-\bar{\alpha}_u +
%1)^{-1}.
%$$
%For $(u)=0$ 
%
%%For $(r) = 0$ and $(u) = 0$ we note that $r \in \bar{I}_0$ and $u \in I_0$ so directly we have $r \neq u$.



%%%%%%%%%%%%%%%%%%%%%%%%%%%%%%%%%%%%%%%%%%%%%%%%%%%%%%%%%%%%%%%%%%%%%%%%%%%%%%%%%%%%%%%
%%%%%%%%%%%%%%%%%%%%%%%%%%%%%%%%%%%%%%%%%%%%%%%%%%%%%%%%%%%%%%%%%%%%%%%%%%%%%%%%%%%%%%%

%\newpage
%\section{Summary of main results} 
%\label{resultsum}

%%%%%%%%%%%%%%%%%%%%%%%%%%%%%%%%%%%%%%%%%%%%%%%%%%%%%%%%%%%%%%%%%%%%%%%%%%%%%%%%%%%%%%%
%%%%%%%%%%%%%%%%%%%%%%%%%%%%%%%%%%%%%%%%%%%%%%%%%%%%%%%%%%%%%%%%%%%%%%%%%%%%%%%%%%%%%%% 


%%%%%%%%%%%%%%%%%%%%%%%%%%%%%%%%%%%%%%%%%%%%%%%%%%%%%%%%%%%%%%%%%%%%%%%%%%%%%%%%%%%%%%%


%%%%%%%%%%%%%%%%%%%%%%%%%%%%%%%%%%%%%%%%%%%%%%%%%%%%%%%%%%%%%%%%%%%%%%%%%%%%%%
%
%  Acknowledgements
%
%%%%%%%%%%%%%%%%%%%%%%%%%%%%%%%%%%%%%%%%%%%%%%%%%%%%%%%%%%%%%%%%%%%%%%%%%%%%%%
%
\section*{Acknowledgments}
%
This work was supported by the Australian Research Council through Discovery Project
DP140101492. 
%
%
%%%%%%%%%%%%%%%%%%%%%%%%%%%%%%%%%%%%%%%%%%%%%%%%%%%%%%%%%%%%%%%%%%%%%%%%%%%%%%
%
%  References
%
%%%%%%%%%%%%%%%%%%%%%%%%%%%%%%%%%%%%%%%%%%%%%%%%%%%%%%%%%%%%%%%%%%%%%%%%%%%%%%
%

\newpage
\begin{thebibliography}{10}

\bibitem{ZhaGou1990} M.D. Gould and R.B. Zhang, J. Math. Phys. {\bf 31} (1990) 2552.

\bibitem{GouZha1990} M.D. Gould and R.B. Zhang, Lett. Math. Phys. {\bf 20} (1990) 221.

\bibitem{SNR1977} M. Scheunert, W. Nahm and V. Rittenberg, J. Math. Phys. {\bf 18} (1977) 146. 
% 
\bibitem{GIW1} M.D. Gould, P.S. Isaac and J.L. Werry, J. Math. Phys. {\bf 54} (2013), 013505.
% 
\bibitem{GIW2} M.D. Gould, P.S. Isaac and J.L. Werry, J. Math. Phys. {\bf 55} (2014), 011703.
%
\bibitem{GIW3} J.L. Werry, M.D. Gould and P.S. Isaac, J. Math. Phys. {\bf 56} (2015), 121703.
% 
\bibitem{Green1971} H.S. Green, J. Math. Phys. {\bf 12} (1971) 2106.
%  
\bibitem{BraGre1971} A.J. Bracken and H.S. Green, J. Math. Phys. {\bf 12} (1971) 2099.
%  
% \bibitem{Green1975} H.S. Green, J. Austral. Math. Soc. Ser. B {\bf 19} (1975) 129.
% 
\bibitem{OBCantCar1977} D.M. O'Brien, A. Cant and A.L. Carey, Ann. Inst. Henri 
Poincar\'e, Section A: Physique th\'eorique {\bf 26} (1977) 405.
  
\bibitem{Gould1985} M.D. Gould, J. Austral. Math. Soc. Ser. B {\bf 26}  (1985) 257.

\bibitem{JarGre1979} P.D. Jarvis and H.S. Green, J. Math. Phys.  {\bf 20}  (1979) 2115.
 
\bibitem{GreJar1983}  H.S. Green and P.D. Jarvis, J. Math. Phys.  {\bf 24}  (1983) 1681.
  
\bibitem{Gould1987}  M.D. Gould, J. Austral. Math. Soc. Ser. B {\bf 28}  (1987) 310.
%  
% \bibitem{GT1950}
% I.M. Gelfand and M.L. Tsetlin, Dokl. Akad. Nauk., SSSR {\bf 71} (1950) 825 (Russian).
% English translation in: I.M. Gelfand, ``Collected Papers'', Vol II, Berlin:
% Springer-Verlag (1988) 653.
%  
% \bibitem{GT1950b}
% I.M. Gelfand and M.L. Tsetlin, Dokl. Akad. Nauk., SSSR {\bf 71} (1950) 1017 (Russian).
% English translation in: I.M. Gelfand, ``Collected Papers'', Vol II, Berlin:
% Springer-Verlag (1988) 657.
%  
% \bibitem{BB1963} G.E. Baird and L.C. Biedenharn, J. Math. Phys. {\bf 4} (1963) 1449.
%  
% \bibitem{Palev1987} T. D. Palev, Funct. Anal. Appl. {\bf 21} (1987) 245.
%  
% \bibitem{Palev1989} T. D. Palev, Funct. Anal. Appl. {\bf 23} (1989) 141.
% 
% \bibitem{StoiVan2010} N.I. Stoilova and J. Van der Jeugt, J. Math. Phys. {\bf 51} (2010)
%  093523.
% 
% \bibitem{Molev2011} A.I. Molev, Bull. Inst. Math. Acad. Sinica {\bf 6} (2011) 415.
% 
% \bibitem{TolIstSmi1986} V.N. Tolstoy, I.F. Istomina and Yu.F. Smirnov, in ``Group
% Theoretical Methods in Physics: Proceedings of the Third Yurmala Seminar'', Yurmala, USSR,
% 1985, Ed. M.A. Markov, V.I. Man'ko and V.V. Dodonov, VNU Science Press, Utrecht (1986)
% 337.
%  
\bibitem{Kac1977} V.G. Kac, Adv. in Math. {\bf 26} (1977) 8.
% 
\bibitem{Kac1978} V.G. Kac, Lecture Notes in Math. {\bf 676}, Springer, Berlin (1978) 597.
% 
\bibitem{NahmSch1976} W. Nahm and M. Scheunert, J. Math. Phys. {\bf 17} (1976) 868.
% 
\bibitem{GouZha19902} M.D. Gould and R.B. Zhang, J. Math. Phys. {\bf 31} (1990) 1524.
% 
\bibitem{GouBraHug1989} M.D. Gould, A.J. Bracken and J.W.B. Hughes, J. Phys. A: Math. Gen.
{\bf 22} (1989) 2879.
 
\bibitem{GouJarBra1990} M.D. Gould, P.D. Jarvis and A.J. Bracken, J. Math. Phys. {\bf 31}
(1990) 2803. 
% 
% \bibitem{PaStVa1994} T.D. Palev, N.I. Stoilova and J. Van der Jeugt, Comm. Math. Phys.
% {\bf 166} (1994) 367.
% 
\bibitem{Gould1992}
M.D. Gould, J. Math. Phys. {\bf 33} (1992) 1023.
% 
\bibitem{Gould1981}
M.D. Gould, J. Math. Phys. {\bf 22} (1981) 15.

% \bibitem{SchNaRit1976} M. Scheunert, W. Nahm and V. Rittenberg, J. Math. Phys. {\bf 17}
% (1976) 1626.
% 
% \bibitem{Sch1979} M. Scheunert, Lecture Notes in Math. {\bf 716}, Springer, Berlin (1979).
% 
% \bibitem{Ram1971} P. Ramond, Phys. Rev. D {\bf 3} (1971) 2415.
% 
% \bibitem{NevSchw1971} A. Neveu and J.H. Schwarz, Nucl. Phys. B {\bf 31} (1971) 86.
% 
% \bibitem{VolAk1973} D.V. Volkov and V.P. Akulov, Phys. Lett. B {\bf 46} (1973) 109.
% 
% \bibitem{WessZum1974} J. Wess and B. Zumino, Nucl. Phys. B {\bf 70} (1974) 39.
% 
% \bibitem{SalStrath1974} A. Salam and J. Strathdee, Nucl. Phys. B {\bf 76} (1974) 477.
% 
% \bibitem{Scherk1975} J. Scherk, Rev. Mod. Phys. {\bf 47} (1975) 123.
% 
% \bibitem{FayFer1977} P. Fayat and S. Ferrara, Phys. Rep. {\bf 32} (1977) 249.
% 
% \bibitem{CorNeSt1975} L. Corwin, Y. Ne'eman and S. Sternberg, Rev. Mod. Phys. {\bf 47}
% (1975) 573.
% 
% \bibitem{Musson2012} I.M. Musson, Graduate Studies in Math. {\bf 131}, AMS (2012).
% 
% \bibitem{BeiStau2003} N. Beisert and M. Staudacher, Nucl. Phys. B {\bf 670} (2003) 439.
% 
% \bibitem{Mina2012} J.A. Minahan, Lett. Math. Phys. {\bf 99} (2012) 33.
% 
% \bibitem{GalMar2004} W. Galleas and M.J. Martins, Nucl. Phys. B {\bf 699} (2004) 455.
% 
% \bibitem{EssFraSal2005} F.H.L. Essler, H. Frahm and H. Saleur, Nucl. Phys. B {\bf 712}
% (2005) 513.
% 
% \bibitem{ZhYaZh2006} S.Y. Zhao, W.L. Yang and Y.Z. Zhang, Commun. Math. Phys. {\bf 268}
% (2006) 505.
% 
% \bibitem{RagSat2007} E. Ragoucy and G. Satta, JHEP09 (2007) 001.
% 
% \bibitem{FraMar2011} H. Frahm and M.J. Martins, Nucl. Phys. B {\bf 847} [FS] (2011) 220.
% 
% \bibitem{SchomSal2006} V. Schomerus and H. Saleur, Nucl. Phys. B {\bf 734} [FS] (2006) 221.
% 
% \bibitem{Ridout2009} D. Ridout, Nucl. Phys. B {\bf 810} [FS] (2009) 503.
% 
% \bibitem{StoiVan2008} N.I. Stoilova and J. Van der Jeugt, J. Phys. A: Math. Theor. {\bf
% 41} (2008) 075202.
% 
% \bibitem{LievStoiVan2008} S. Lievens, N.I. Stoilova and J. Van der Jeugt, Commun. Math.
% Phys. {\bf 281} (2008) 805.
% 
% 
% \bibitem{Molev2006} A.I. Molev, Handbook of Algebra {\bf 4} (2006) 109.
 
% 
% \bibitem{KamKyPal1989} A.H. Kamupingene, N.A. Ky, T.D. Palev, J. Math. Phys. {\bf 30}
% (1989) 553.
% 
% 
% \bibitem{PalStoi1990} T.D. Palev, N.I. Stoilova, J. Math. Phys. {\bf 31} (1990) 953.
% 
% 
% \bibitem{Dirac1936} P.A.M. Dirac, Proc. R. Soc. Lond. A {\bf 155} (1936) 447.
% 
% \bibitem{BB1964b} G.E. Baird and L.C. Biedenharn, J. Math. Phys. {\bf 5} (1964) 1723.
% 
% \bibitem{Gould1978}
% M.D. Gould, J. Austral. Math. Soc. Ser. B {\bf 20}  (1978) 401.
% 
% \bibitem{Gould1980}
% M.D. Gould, J. Math. Phys. {\bf 21} (1980) 444.
% 
% 
% \bibitem{Gould1981b}
% M.D. Gould, J. Math. Phys. {\bf 22} (1981) 2376.
% 
% \bibitem{Gould1984} M.D. Gould, J. Phys. A: Math. Gen. {\bf 17} (1984) 1.
% 
% \bibitem{Kostant1975} B. Kostant, J. Func. Anal {\bf 20} (1975) 257.
% 
% \bibitem{Hannabuss1997} K. Hannabuss, ``An Introduction to Quantum Theory'', Oxford
% University Press, Oxford (1997).
% 
% \bibitem{BB1964} G.E. Baird and L.C. Biedenharn, J. Math. Phys. {\bf 5} (1964) 1730.
% 
% \bibitem{LouBei1970} J.D. Louck and L.C. Biedenharn, J. Math. Phys. {\bf 11} (1970) 2368.
% 
% \bibitem{ZhGoBr1991} R.B. Zhang, M.D. Gould and A.J. Bracken, Nucl. Phys. B {\bf 354}
% (1991) 625.
% 
% \bibitem{RitSch1992} V. Rittenberg and M. Scheunert, J. Math. Phys. {\bf 33} (1992) 436.
% 
% \bibitem{Mozr2005} M. Mozrzymas, Int. J. Geom. Meth. Mod. Phys. {\bf 2} (2003) 393.
% 
% \bibitem{PaisRitt1975} A. Pais and V. Rittenberg, J. Math. Phys. {\bf 16} (1975) 2062.
% 
% \bibitem{Mezin1977} L. Mezincescu, J. Math. Phys. {\bf 18} (1977) 453.
% 
% \bibitem{Mozr2004} M. Mozrzymas, J. Phys. A: Math. Gen. {\bf 37} (2004) 9515.
% 
% \bibitem{JarMur1983} P.D. Jarvis and M.K. Murray, J. Math. Phys. {\bf 24} (1983) 1705.
% 
% \bibitem{JaRuYa2011}
% P.D. Jarvis, G. Rudolph and L.A. Yates, J. Phys. A: Math. Theor. {\bf 44} (2011) 235205.

\end{thebibliography}


\end{document}
